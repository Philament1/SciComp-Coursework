%Template file for Scientific Computation project 3 discussion and figures
\documentclass{article}
\usepackage[a4paper, margin=1in]{geometry}
\title{Scientific Computation Project 3}
\usepackage{minted}
\usepackage{amsmath}
\usepackage{amssymb}
\usepackage{placeins}
\newcommand{\trm}{\textrm}
\newcommand{\pa}{\partial}
\author{\emph{Your college ID here}}

\usepackage{graphicx}

\begin{document}

\maketitle

%---------------- Part 1  -------------------
\hrule
\hrule

\subsection*{Part 1}
%place discussion for Part 1 here

We begin by reshaping $u$ into a $2304 \times 365$ \textit{NumPy} array, with each day as a datapoint in the columns, and locations (longitude and latitude combinations) as the attributes in the rows, and we apply PCA to it.

When applying PCA, we note that the number of non-zero singular values is 364, which shows that the matrix $A$ is not full rank as we have 365 days. 

\begin{figure}[h!]
\centering
\includegraphics[width=\textwidth]{fig_pc1.png}
\caption{First 3 PCs}
\label{fig_pc1}
\end{figure}

Figure \ref{fig_pc1} shows the first ?? principal components, i.e. the directions in which the highest variance in wind speed is captured.

\begin{figure}[h!]
\centering
\includegraphics[width=0.5\textwidth]{fig_var1.png}
\caption{Cumulative retained variance of PCs}
\label{fig_var1}
\end{figure}

From Figure \ref{fig_var1}, we observe that 80\% of the total variance is retained at around ?? principal components. 

\begin{figure}[h!]
\centering
\includegraphics[width=0.8\textwidth]{fig_proj1c.png}
\caption{FFT and Welch's on projection onto PC1}
\label{fig_proj1c}
\end{figure}

From Figure \ref{fig_proj1c}, we see there are no dominant frequencies in both spectral density estimates. The frequency density also appears to decrease as frequency increases, indicating that trends appear to be gradual and seasonal, and there are fewer high-frequency variations, such as rapid changes in windspeed. 

\begin{figure}[h!]
\centering
\includegraphics[width=0.8\textwidth]{fig_proj1.png}
\caption{Subsequent days of PC1 and PC2}
\label{fig_proj1b}
\end{figure}

We see from Figure \ref{fig_proj1} that there is a general positive correlation in wind speed between subsequent days. 

We then transpose our array into a $365 \times 2304$ \textit{NumPy} array, now treating the locations as the datapoints, and each day as an attribute. 

\begin{figure}[h!]
\centering
\includegraphics[width=0.8\textwidth]{fig_pc2.png}
\caption{First 2 Principal Components}
\label{fig_pc2}
\end{figure}

Figure \ref{fig_pc2} shows the first 2 principal components, i.e. the values in time that capture the most variance. 

\begin{figure}[h!]
\centering
\includegraphics[width=0.5\textwidth]{fig_var2.png}
\caption{Cumulative retained variance of PCs}
\label{fig_var2}
\end{figure}

From Figure \ref{fig_var2}, we observe that 80\% of the total variance is retained at around ?? principal components. 

\begin{figure}[h!]
\centering
\includegraphics[width=\textwidth]{fig_proj2.png}
\caption{Projected values of first 3 principal components}
\label{fig_proj2}
\end{figure}

From \ref{fig_proj2}, we see a large variation in scores between the higher and lower latitudes, indicating great differences in windspeeds between the upper and lower halves of the map. The second and third principal components also show clusters of locations where the scores are similar, showing patterns in space.

%---------------- End Part 1 -------------------

\FloatBarrier
\vspace{0.25in}

%---------------- Part 2  -------------------
\subsection*{Part 2}

\subsubsection*{2.}
%Place your discussion for question 2 here}
For method 2, we solve the system of linear equations
\[
    \mathbf{A\tilde{f}} = \mathbf{Bf}
\]
for $\mathbf{\tilde{f}}$ with columns $\mathbf{\tilde{f}}_j = (\tilde{f}_{1/2,j}, \dots, \tilde{f}_{m-3/2,j})^\top$, where
\[
    \mathbf{A} = 
    \begin{pmatrix}
        1 \\
        \alpha & 1 & \alpha \\
        & \ddots & \ddots & \ddots \\
        && \alpha & 1 & \alpha \\
        &&&& 1
    \end{pmatrix}
\]
\[
    \mathbf{B} = 
    \begin{pmatrix}
        \tilde{a} & \tilde{b} & \tilde{c} & \tilde{d} \\
        \frac{b}{2} & \frac{a}{2} & \frac{a}{2} & \frac{b}{2} \\
        & \ddots & \ddots & \ddots & \ddots\\
        && \frac{b}{2} & \frac{a}{2} & \frac{a}{2} & \frac{b}{2} \\
        && \tilde{d} & \tilde{c} & \tilde{b} & \tilde{a}
    \end{pmatrix}
\]
and the columns of $\mathbf{f}$ are $\mathbf{f}_j = (f_{0,j}, \dots, f_{m-1,j})^\top$. 

To solve this linear equation efficiently, we convert matrix $\mathbf{A}$ to ``matrix diagonal ordered form'' and then apply \textit{scipy.linalg.solve\_banded}.

We run both methods on the following functions:
\begin{itemize}
    \item \textit{Oscillating function}
    \[
        f(x, y) = \sin(2\pi x)\cos(2\pi y)
    \]
    \item \textit{Franke's function - used as a test function for interpolation problems}
    \begin{align*}
        f(x, y) = & 0.75 \exp\left(-\frac{(9x - 2)^2}{4} - \frac{(9y - 2)^2}{4}\right) + 0.75 \exp\left(-\frac{(9x + 1)^2}{49} - \frac{(9y + 1)}{10}\right) \\
        & + 0.5 \exp\left(-\frac{(9x - 7)^2}{4} - \frac{(9y - 3)^2}{4}\right) - 0.2 \exp\left(-\frac{(9x - 4)^2}{49} - \frac{(9y - 7)^2}{10}\right)
    \end{align*}
    
\end{itemize}

To determine the accuracy, cost and efficiency of both methods, we apply them to each function and measure the average wall times (over 1000 runs) and RMSE (Root Mean Square Error). 

\begin{table}
  \centering
  \begin{tabular}{|c|c|c|}
    \hline
     & RMSE Method 1 & RMSE Method 2 \\
    \hline
    Oscillating function & 2.5761873814328875e-06 & 1.8641643487025784e-11 \\
    Franke's function & 4.884560478922877e-07 & 9.545519469236035e-12 \\
    \hline
  \end{tabular}
  \caption{RMSE}
  \label{tab:part2_RMSE}
\end{table}

\begin{table}
  \centering
  \begin{tabular}{|c|c|c|}
    \hline
     & Mean Wall Time (s) Method 1 & Mean Wall Time (s) Method 2 \\
    \hline
    Oscillating function & 7.0264339447021485e-06 & 0.0002693333625793457 \\
    Franke's function & 6.220340728759766e-06 & 0.00034125757217407226 \\
    \hline
  \end{tabular}
  \caption{Wall Times}
  \label{tab:part2_walltimes}
\end{table}

\begin{figure}[h!]
\centering
%Uncomment line below to display figure saved as fig2.png
%\includegraphics[width=0.8\textwidth]{fig2.png}

\caption{Add figure description here}
\label{fig2}
\end{figure}

%Add additional figure if needed
%---------------- End Part 2 -------------------
\FloatBarrier
%---------------- Part 3  -------------------
\subsection*{Part 3}

\subsubsection*{1.}
%Place your discussion for question 1 here}


\subsubsection*{2.}
%Place your brief discussion for question 2 here}
The 4 lines of code in function \textit{part3q2} apply principal component analysis to the solution matrix $\mathbf{A}$, then return a lower dimensional approximate reconstruction of $\mathbf{A}$, along with the error of the reconstruction. 
\begin{enumerate}
    \item The first line retrieves the principal components of $\mathbf{A}$ efficiently by finding $\mathbf{v_1}$, the eigenvectors of the covariance matrix $\mathbf{A}^\top\mathbf{A}$, using \textit{np.linalg.eigh}. Calculating the covariance matrix and its eigenvalues is computationally faster than the SVD algorithms we use in \textit{part1} for PCA, however, it is more memory-intensive and can be numerically unstable.
    \item The second line calculates $\mathbf{v_2}$, the transformed data matrix, by transforming $\mathbf{A}$ into the space spanned by the principal components using matrix multiplication.
    \item The third line uses the first $x$ columns of the transformed data matrix, along with the first $x$ principal components, to reconstruct the $x$-dimensional approximation of $A$. This again uses matrix multiplication
    \item The last line calculates the square error of the reconstruction efficiently using \textit{NumPy} vectorised operations.
\end{enumerate}   

\begin{figure}[h!]
\centering
%Uncomment line below to display figure saved as fig3.png
%\includegraphics[width=0.8\textwidth]{fig3.png}

\caption{Add figure description here}
\label{fig3}
\end{figure}

%---------------- End Part 3 -------------------
%Add additional figures as needed

\hrule
\hrule



%---------------- End document -------------------


\end{document}
