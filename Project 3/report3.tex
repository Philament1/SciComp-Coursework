%Template file for Scientific Computation project 3 discussion and figures
\documentclass{article}
\usepackage[a4paper, margin=1in]{geometry}
\title{Scientific Computation Project 3}
\usepackage{minted}
\usepackage{amsmath}
\usepackage{amssymb}
\usepackage{placeins}
\newcommand{\trm}{\textrm}
\newcommand{\pa}{\partial}
\author{\emph{02027072}}

\usepackage{graphicx}

\begin{document}

\maketitle

%---------------- Part 1  -------------------
\hrule
\hrule

\subsection*{Part 1}
%place discussion for Part 1 here

We begin by reshaping $u$ into a $2304 \times 365$ \textit{NumPy} array, with each day as a datapoint in the columns, and locations (longitude and latitude combinations) as the attributes in the rows, and we apply PCA to it.

When applying PCA, we note that the number of non-zero singular values is 364, which shows that the matrix $A$ is not full rank as we have 365 days. 

\begin{figure}[h!]
\centering
\includegraphics[width=\textwidth]{fig_pc1.png}
\caption{First 3 PCs}
\label{fig_pc1}
\end{figure}

Figure \ref{fig_pc1} shows the first 3 principal components, i.e. the directions in which the highest variance in wind speed is captured. The maps show distinct zones which exhibit similar behaviour over time. 

\begin{figure}[h!]
    \begin{minipage}{0.49\textwidth}
    \centering
    \includegraphics[width=\textwidth]{fig_var1.png}
    \caption{Cumulative retained variance of PCs}
    \label{fig_var1}
    \end{minipage}%
    \hfill
    \begin{minipage}{0.49\textwidth}
    \centering
    \includegraphics[width=\textwidth]{fig_proj1.png}
    \caption{Projections onto first 2 PCs}
    \label{fig_proj1}
    \end{minipage}%
\end{figure}

From Figure \ref{fig_var1}, we observe that 80\% of the total variance is retained at around 20 principal components and that under 20\% of the total variance is captured by the first PC. 

Figure \ref{fig_proj1} shows our transformed data values as projections onto the first 2 principal components. There are no clear patterns in this so we do further analysis on this time series. 

\begin{figure}[h!]
    \begin{minipage}{0.49\textwidth}
    \centering
    \includegraphics[width=\textwidth]{fig_proj1b.png}
    \caption{FFT and Welch's on projection onto PC1}
    \label{fig_proj1b}
    \end{minipage}
    \hfill
    \begin{minipage}{0.49\textwidth}
    \centering
    \includegraphics[width=0.8\textwidth]{fig_proj1c.png}
    \caption{Subsequent days of PC1 and PC2}
    \label{fig_proj1c}
    \end{minipage}%
\end{figure}

From Figure \ref{fig_proj1b}, we observe there are none of the frequencies significantly dominate the others in both spectral density estimates, suggesting there are no consistent periodic patterns in wind speed over the year. The frequency density also appears to decrease as frequency increases, indicating that trends appear to be gradual and seasonal, and there are fewer high-frequency variations, such as rapid changes in windspeed. 

We see from Figure \ref{fig_proj1c} that there is a general positive correlation in wind speed between subsequent days, supporting our idea that there are few high-frequency variations. 

\FloatBarrier

We then transpose our array into a $365 \times 2304$ \textit{NumPy} array, now treating the locations as the datapoints, and each day as an attribute. 

Figure \ref{fig_pc2} shows the first 2 principal components, i.e. the values in time that capture the most variance. 

\begin{figure}[h!]
    \begin{minipage}{0.49\textwidth}
    \centering
    \includegraphics[width=0.8\textwidth]{fig_pc2.png}
    \caption{First 2 Principal Components}
    \label{fig_pc2}
    \end{minipage}
    \hfill
    \begin{minipage}{0.49\textwidth}
    \centering
    \includegraphics[width=\textwidth]{fig_var2.png}
    \caption{Cumulative retained variance of PCs}
    \label{fig_var2}
    \end{minipage}
\end{figure}

From Figure \ref{fig_var2}, we observe that 80\% of the total variance is retained at around 10 principal components, and that the first principal component captures over 50\% of the total variance. 

\begin{figure}[h!]
\centering
\includegraphics[width=\textwidth]{fig_proj2.png}
\caption{Projected values of first 2 principal components}
\label{fig_proj2}
\end{figure}

From Figure \ref{fig_proj2}, we see a large variation in scores between the higher and lower latitudes, indicating great differences in windspeed between the upper and lower halves of the map. The second principal component also shows distinct zones where the scores are similar, indicating the presence of spatial clusters and patterns. 

%---------------- End Part 1 -------------------

\FloatBarrier
\vspace{0.25in}

%---------------- Part 2  -------------------
\subsection*{Part 2}

\subsubsection*{2.}
%Place your discussion for question 2 here}
For method 2, we solve the system of linear equations
\[
    \mathbf{A\tilde{f}} = \mathbf{Bf}
\]
for $\mathbf{\tilde{f}}$ with columns $\mathbf{\tilde{f}}_j = (\tilde{f}_{1/2,j}, \dots, \tilde{f}_{m-3/2,j})^\top$, where
\[
    \mathbf{A} = 
    \begin{pmatrix}
        1 \\
        \alpha & 1 & \alpha \\
        & \ddots & \ddots & \ddots \\
        && \alpha & 1 & \alpha \\
        &&&& 1
    \end{pmatrix}
\]
is a banded square matrix,
\[
    \mathbf{B} = 
    \begin{pmatrix}
        \tilde{a} & \tilde{b} & \tilde{c} & \tilde{d} \\
        \frac{b}{2} & \frac{a}{2} & \frac{a}{2} & \frac{b}{2} \\
        & \ddots & \ddots & \ddots & \ddots\\
        && \frac{b}{2} & \frac{a}{2} & \frac{a}{2} & \frac{b}{2} \\
        && \tilde{d} & \tilde{c} & \tilde{b} & \tilde{a}
    \end{pmatrix}
\]
is a rectangular banded matrix, and the columns of $\mathbf{f}$ are $\mathbf{f}_j = (f_{0,j}, \dots, f_{m-1,j})^\top$. 

To solve this linear equation efficiently, we convert matrix $\mathbf{A}$ to ``matrix diagonal ordered form'' and then apply \textit{scipy.linalg.solve\_banded}.

We apply both methods to the following functions on a 2-dimensional grid:
\begin{itemize}
    \item \textit{Oscillating function}
    \[
        f(x, y) = \sin(2\pi x)\cos(2\pi y)
    \]
    \item \textit{Franke's function - used as a test function for interpolation problems}
    \begin{align*}
        f(x, y) = & 0.75 \exp\left(-\frac{(9x - 2)^2}{4} - \frac{(9y - 2)^2}{4}\right) + 0.75 \exp\left(-\frac{(9x + 1)^2}{49} - \frac{(9y + 1)}{10}\right) \\
        & + 0.5 \exp\left(-\frac{(9x - 7)^2}{4} - \frac{(9y - 3)^2}{4}\right) - 0.2 \exp\left(-\frac{(9x - 4)^2}{49} - \frac{(9y - 7)^2}{10}\right)
    \end{align*}
    
\end{itemize}

To determine the accuracy, cost and efficiency of both methods, we first apply them to each function and plot the errors as the absolute difference between the interpolated value and the actual value. We also measure the average wall times (over 1000 runs) and calculate the mean error (mean absolute difference between interpolated and calculated values). We test the function on a grid with sizes $n=50$ and $m=40$.

\FloatBarrier

\begin{figure}[h!]
\centering
\includegraphics[width=0.8\textwidth]{fig_osc.png}
\caption{Results for oscillating function}
\label{fig_osc}
\end{figure}

\begin{figure}[h!]
\centering
\includegraphics[width=0.8\textwidth]{fig_franke.png}
\caption{Results for Franke's function}
\label{fig_franke}
\end{figure}

From figures \ref{fig_osc} and \ref{fig_franke}, we observe that method 1 produces large errors at the peaks and troughs of both functions. Method 2 has a much more stable error value throughout both functions, except at the boundaries, where there is a significantly larger error value.

\begin{table}[h!]
  \centering
  \begin{tabular}{|c|c|c|}
    \hline
     & Error Method 1 & Error Method 2 \\
    \hline
    Oscillating function & 2.5761873814328875e-06 & 1.8641643487025784e-11 \\
    Franke's function & 4.884560478922877e-07 & 9.545519469236035e-12 \\
    \hline
  \end{tabular}
  \caption{Mean Error}
  \label{tab:part2_err}
\end{table}

From Table \ref{tab:part2_err}, we note that method 2 has a significantly lower error for both functions, hence it is more accurate overall despite the higher error on the boundaries.

\begin{table}[h!]
  \centering
  \begin{tabular}{|c|c|c|}
    \hline
     & Mean Wall Time (s) Method 1 & Mean Wall Time (s) Method 2 \\
    \hline
    Oscillating function & 7.0264339447021485e-06 & 0.0002693333625793457 \\
    Franke's function & 6.220340728759766e-06 & 0.00034125757217407226 \\
    \hline
  \end{tabular}
  \caption{Wall Times}
  \label{tab:part2_walltimes}
\end{table}

From Table \ref{tab:part2_walltimes}, we see that method 2 has significantly longer wall times, as the method has a higher computational cost due to the use of matrix multiplication and solving linear equations, in comparison to the simpler method 1.

\FloatBarrier

We then test both methods on Franke's function for a range of different grid sizes (always letting $n=m+10$ for consistency).

\begin{figure}[h!]
\centering
\includegraphics[width=0.7\textwidth]{fig_yerror.png}
\caption{Mean Error for each $y$}
\label{fig_yerror}
\end{figure}

In Figure \ref{fig_yerror}, we test $m=40, 80, \dots, 160, 200$. We observe that the error is more consistent for different $y$ with method 1, whereas method 2 is significantly worse at the boundaries. Both methods are more accurate at higher $m$ values, and method 2 is consistently more accurate than method 1.

\begin{figure}[h!]
\centering
\includegraphics[width=0.7\textwidth]{fig_merror.png}
\caption{Mean Error against $m$}
\label{fig_merror}
\end{figure}

In Figure \ref{fig_merror}, we test for higher $m$ values. The error for increasing $y$ is almost exponential for both methods, with the error for method 2 decreasing at a higher rate than method 1.

%Add additional figure if needed
%---------------- End Part 2 -------------------
\FloatBarrier
%---------------- Part 3  -------------------
\subsection*{Part 3}

\subsubsection*{1.}
%Place your discussion for question 1 here}
For this question, we perform analysis on $c=0.5, 1.2, 1.3, 1.5$, and we let $x_i=u_{i-100}$ for $i = 100$ to $n-100$. 

\begin{figure}
  \centering

  \begin{subfigure}{0.45\textwidth}
    \includegraphics[width=\linewidth]{image1.png}
    \caption{Caption for Image 1}
    \label{fig:sub1}
  \end{subfigure}
  \hfill
  \begin{subfigure}{0.45\textwidth}
    \includegraphics[width=\linewidth]{image2.png}
    \caption{Caption for Image 2}
    \label{fig:sub2}
  \end{subfigure}

  \medskip

  \begin{subfigure}{0.45\textwidth}
    \includegraphics[width=\linewidth]{image3.png}
    \caption{Caption for Image 3}
    \label{fig:sub3}
  \end{subfigure}
  \hfill
  \begin{subfigure}{0.45\textwidth}
    \includegraphics[width=\linewidth]{image4.png}
    \caption{Caption for Image 4}
    \label{fig:sub4}
  \end{subfigure}

  \caption{Overall caption for the 2x2 figure.}
  \label{fig:overall}
\end{figure}


\subsubsection*{2.}
%Place your brief discussion for question 2 here}
The 4 lines of code in function \textit{part3q2} apply principal component analysis to the solution matrix $\mathbf{A}$, then return a lower $x$-dimensional approximate reconstruction of $\mathbf{A}$, along with the error of the reconstruction. 
\begin{enumerate}
    \item The first line retrieves the principal components of $\mathbf{A}$ efficiently by finding $\mathbf{v_1}$, the eigenvectors of the covariance matrix $\mathbf{A}^\top\mathbf{A}$, using \textit{np.linalg.eigh}. Calculating the covariance matrix and its eigenvalues is computationally faster than the SVD algorithms we use in \textit{part1} for PCA, however, it is more memory-intensive and can be numerically unstable.
    \item The second line calculates $\mathbf{v_2}$, the transformed data matrix, by transforming $\mathbf{A}$ into the space spanned by the principal components using matrix multiplication.
    \item The third line uses the first $x$ columns of the transformed data matrix, along with the first $x$ principal components, to reconstruct the rank-$x$ approximation of $A$. This again uses matrix multiplication
    \item The last line calculates the square error of the reconstruction efficiently using \textit{NumPy} vectorised operations.
\end{enumerate}   

\begin{figure}[h!]
\centering
%Uncomment line below to display figure saved as fig3.png
%\includegraphics[width=0.8\textwidth]{fig3.png}

\caption{Add figure description here}
\label{fig3}
\end{figure}

%---------------- End Part 3 -------------------
%Add additional figures as needed

\hrule
\hrule



%---------------- End document -------------------


\end{document}
