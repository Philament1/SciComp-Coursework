%Template file for Scientific Computation project 1 discussion and figures
\documentclass{article}
\usepackage[a4paper, margin=1in]{geometry}
\title{Scientific Computation Project 1}
\usepackage{minted}
\usepackage{amsmath}
\usepackage{amssymb}
\newcommand{\trm}{\textrm}
\newcommand{\pa}{\partial}
\author{\emph{02027072}}

\usepackage{graphicx}

\begin{document}

\maketitle

%---------------- Part 1  -------------------
\hrule
\hrule

\subsection*{Part 1}


\subsubsection*{1.}
%Place your discussion for question 1 here
The strategy \textit{part1} uses a hybrid sorting algorithm, where initially the elements of the list are sorted using an insertion sort, up to the index $istar$. Then, a variation of the insertion using a binary search determines where to insert the subsequent elements with an index greater than $istar$. 

The worst-case scenario for an insertion sort of length $n$ is when the list is reversed, and by lecture slides, this has an $\mathcal{O}(n^2)$ time complexity. Hence for $i \leq istar$, the worst-case computational cost is $\mathcal{O}(istar^2)$. 

For the case where $i > istar$, a binary search is performed on the elements up to $i$ to determine its location in the sorted list. The worst-case comparison process in a binary search of size $i$ has a time complexity of $\mathcol{O}(\log(i))$ as seen by lecture slides, and since this process is done for $i$ from $istar+1$ up to $N-1$, the worst-case computational cost of this part of the algorithm is $\mathcol{O}(N\log(N))$.

Combining both parts of the sorting algorithm means the worst-case computational cost for the algorithm overall is $O(istar^2 + N\log(N)$. By this conclusion, setting $istar$ to a smaller value would minimise computational cost.

\subsubsection*{2.}
%Place your discussion for question 2 here


\begin{figure}[h!]
\centering
%Uncomment line below to display figure saved as fig1.png
\includegraphics[width=0.8\textwidth]{fig1.png}

\caption{Figure 1: Add figure description here}
\label{fig1}
\end{figure}

%Add additional figure if needed



%---------------- End Part 1 -------------------

\vspace{0.25in}

%---------------- Part 2  -------------------
\subsection*{Part 2}

\subsubsection*{2.}

%Place your discussion for question 2 here}
The method I use i

%---------------- End Part 2 -------------------


\hrule
\hrule



%---------------- End document -------------------


\end{document}
