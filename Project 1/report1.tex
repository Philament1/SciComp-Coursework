%Template file for Scientific Computation project 1 discussion and figures
\documentclass{article}
\usepackage[a4paper, margin=1in]{geometry}
\title{Scientific Computation Project 1}
\usepackage{minted}
\usepackage{amsmath}
\usepackage{amssymb}
\newcommand{\trm}{\textrm}
\newcommand{\pa}{\partial}
\author{\emph{02027072}}

\usepackage{graphicx}

\begin{document}

\maketitle

%---------------- Part 1  -------------------
\hrule
\hrule

\subsection*{Part 1}


\subsubsection*{1.}
%Place your discussion for question 1 here
The strategy \textit{part1} uses is a hybrid sorting algorithm, where initially the elements of the list are sorted using an insertion sort for an index $i \leq istar$. Then, a variation of the insertion using a binary search to replace the reverse linear search determines where to insert the subsequent elements with an index $i>istar$.

The worst-case scenario for an insertion sort of length $n$ is when the list is sorted in reverse, and by lecture slides, this has an $\mathcal{O}(n^2)$ time complexity. Hence for $i \leq istar$, the worst-case computational cost is $\mathcal{O}(istar^2)$. 

For the case where $i > istar$, a binary search is performed on the elements up to $i$ to determine its location in the sorted list of size $i$. The worst-case scenario for a binary search of size $i$ has a time complexity of $\mathcal{O}(\log(i))$ as seen by lecture slides. This search is performed on $i$ from $istar+1$ up to $N-1$, so the overall complexity is  $\log(istar + 1) + \log(istar + 2) \ldots + \log(N-1)$. Hence the worst-case computational cost of this part of the algorithm is $\mathcal{O}(N\log(N))$.

Combining both parts of the sorting algorithm means the worst-case computational cost for the algorithm overall is $\mathcal{O}(istar^2 + N\log(N))$, and choosing $istar=N-1$ would lead to the worst-case computational cost of $\mathcal{O}(N^2)$. By this conclusion, setting $istar=0$ would minimise the computational cost to $\mathcal{O}(N\log(N))$ for the worst-case input i.e. when the list is sorted in reverse (as this is also a worst-case scenario for binary search). 

For $i \leq istar$, the best-case scenario would be a list sorted in ascending order, and the complexity would be $\mathcal{O}(istar)$. For the remaining $i$ up to $N-1$, the best-case scenario would require a specifically designed list where every subsequent element only required one comparison to find its correct index at the halfway point using the binary search, hence the complexity would be $\mathcal{O}(N)$. So overall, the best-case computational cost would be $\mathcal{O}(istar + N)$.

\subsubsection*{2.}
%Place your discussion for question 2 here
In all four plots, we see a clear positive trend between time and N. For this section, we use the words `ascending' and `non-decreasing' interchangeably, as well as `descending' and `non-increasing'.

In the first plot, we observe that for a list in descending order, the lower the $istar$ the faster the wall time. This is expected as the binary search part of the algorithm is faster than the reverse linear search for the worst-case scenario. For $istar=N-1$, we can also see the timings resemble a quadratic trend, which supports our conclusion of the $\mathcal{O}(N^2)$ complexity.

In the second plot, we observe that for a list in ascending order, the greater the $istar$ the faster the wall time. This is expected as for a list in ascending order, the reverse linear search part of the algorithm has a complexity of $\mathcal{O}(1)$ compared to the slower complexity of the binary search $\mathcal{O}(log(N))$. 

In the third plot, we observe that for $istar=0$ (entirely using a binary search modified insertion sort), plotting time against $N\log N$ exhibits an almost linear positive relationship for ascending and descending cases, which is what we expect since both have a complexity of $\mathcal{O}(N)$. We also observe that a sorted list in ascending order is faster than a random list and a list in descending order. This is possibly due to the list assignment or swapping part of the algorithm, as for a list in reverse order this part adds extra time to the algorithm, but for a list in ascending order, this assignment is skipped.

In the fourth plot, we observe that for $istar=N-1$ (entirely using a classic insertion sort), plotting time against $N^2$ exhibits an almost linear positive relationship for a list in descending order and a random list. This supports our calculation of a $\mathcal{O}(N^2)$ worst-case time complexity calculation. We also observe that an ascending list is significantly faster than a random list, which is faster than a descending list. 

\begin{figure}[h!]
\centering
%Uncomment line below to display figure saved as fig1.png
\includegraphics[width=0.8\textwidth]{fig1.png}

\caption{Figure 1: Add figure description here}
\label{fig1}
\end{figure}

%Add additional figure if needed



%---------------- End Part 1 -------------------

\vspace{0.25in}

%---------------- Part 2  -------------------
\subsection*{Part 2}

\subsubsection*{2.}

%Place your discussion for question 2 here}
For \textit{part2}, the strategy I used was the Rabin Karp algorithm to compare each length-m sub-string in T to each length-m sub-string in S. I implemented the \textit{char2base4} and \textit{heval} functions from lecture slides to calculate the two initial base hashes for the first length-m sub-string of S and of T and perform a comparison between them. I then implemented the first Rabin Karp algorithm to calculate a hash using a rolling hash function for every length-m sub-string of T. Each hash is stored as the key in a dictionary, with the value being a list of indexes indicating the locations of each hash in T. I also compare each hash to the base hash of S. 

With everything initialised, I implemented the second Rabin Karp algorithm. Using a rolling hash function, a hash is calculated for every length-m sub-string of S, and then checked against the T hash dictionary. If found, every index in the dictionary value is then checked with a string comparison and appended to the corresponding list in L if there's a match. The string comparison is to protect against hash collisions.

In the worst-case scenario, there are many hash collisions i.e. the dictionary has few unique keys with long lists of indexes. Looping through each sub-string of S has a complexity of $\mathcal{O}(n)$, looping through each T hash dictionary list of indexes would have a complexity of $\mathcal{O}(l)$, and we assume the complexity of a string comparison is $\mathcal{O}(m)$, hence my algorithm would have computational cost of $\mathcal{O}(lmn)$. This would be equivalent to the naive case of double looping through every length-m sub-string of S and T and performing a character-by-character check. However, this case is unlikely for large $l$, $m$, and $n$, and by choosing a large prime, fewer hash collisions will occur (at the expense of memory usage).

In the best-case scenario, there would be no hash matches, so the lists in the T hash dictionary and string comparisons would not have to occur, and my algorithm would have a computational cost of $\mathcal{O}(l + n)$ for simply calculating each hash value.

The use of a dictionary over a list to store the hashes for T is to make the hash-matching process more efficient. The cost of a dictionary lookup is $\mathcal{O}(1)$, compared to iterating through a list of hashes for T which has a cost of $\mathcal{O}(l)$.
%---------------- End Part 2 -------------------


\hrule
\hrule



%---------------- End document -------------------


\end{document}
